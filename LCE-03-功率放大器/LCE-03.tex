% 若编译失败,且生成 .synctex(busy) 辅助文件,可能有两个原因:
% 1. 需要插入的图片不存在:Ctrl + F 搜索 'figure' 将这些代码注释/删除掉即可
% 2. 路径/文件名含中文或空格:更改路径/文件名即可

% --------------------- 文章宏包及相关设置 --------------------- %
% >> ------------------ 文章宏包及相关设置 ------------------ << %
% 设定文章类型与编码格式
\documentclass[UTF8]{article}		
\input{../.config/config_for_LinearCircuitExperiment.tex}



%%%%%%%%%%%%%%%%%%%%%%%%%%%%%%%%%%%%%%%%%%%%%%%%%%%%%%%%%%%%%%%%
% 仅需修改页眉、实验名称、实验日期
%%%%%%%%%%%%%%%%%%%%%%%%%%%%%%%%%%%%%%%%%%%%%%%%%%%%%%%%%%%%%%%%


%%%%%%%%%%%%%%%%%% 1. 修改页眉内容 %%%%%%%%%%%%%%%%%%
\rhead{LCE-03 功率放大器 (2025.04.18, 丁毅)}

% 开始编辑文章
\begin{document}
\begin{center}\large
    \vspace*{-0.8cm}
    \noindent{\huge\bfseries《\ \ 线\ \ 性\ \ 电\ \ 路\ \ 实\ \ 验\ \ \ 》\ \ 实\ \ 验\ \ 报\ \ 告 }
    \\\vspace{0.1cm}
    \noindent{
    {\bfseries 
%
%%%%%%%%%%%%%%%%%% 2. 修改实验名称 %%%%%%%%%%%%%%%%%%
    实验名称:\uline{\hspace{1.8cm} 功率放大器 \hspace{1.8cm}}
%
    }\hspace{0.4cm}
    指导教师:\uline{\hspace{0.8cm}王东雷\ \ \ \  \ df4dac@sina.com     \hspace{0.8cm}}
    }
    \\\vspace{0.1cm}
    \noindent
    {
    姓名:\uline{\,\,\,丁毅\,\,\,}\hspace{0.2cm}
    学号:\uline{\,\,\,{ 2023K8009908031}\,\,\,}\hspace{0.2cm}
    班级/专业:\uline{\,\,\,{2308/电子信息}\,\,\,}\hspace{0.2cm}
    分组序号:\uline{\,\,\,{2-06}\,\,\,}
    }
    \\\vspace{0.1cm}
    \noindent{
%
%%%%%%%%%%%%%%%%%% 3. 修改实验日期 %%%%%%%%%%%%%%%%%%
    实验日期:\uline{\,\,{2025.04.18}\,\,}\hspace{0.2cm}
%
    实验地点:\uline{\,\,\,教学楼{ 607}\,\,\,}\hspace{0.2cm}
    是否调课/补课:\uline{\hspace{0.26cm}否 \hspace{0.26cm}}\hspace{0.2cm}
    成绩:\uline{\hspace{2cm}}
    }
\end{center}
\vspace{-0.4cm}
\noindent\rule{\textwidth}{0.075em}   % 分割线
\vspace{-1.0cm}

% 生成目录
\setcounter{tocdepth}{3}  % 目录深度为 2(不显示 subsubsection)
\noindent\tableofcontents\thispagestyle{fancy}   % 显示页码、页眉等

% ------------------------ 文章信息区 ------------------------ %
% ------------------------ 文章信息区 ------------------------ %



%%%%%%%%%%%%%%%%%%%%%%%%%%%%%%%%%%%%%%%%%%%%%%%%%%%%%%%%%%%%%%%%%%%%%%%%%%%%%%%%%
%%%%%%%%%%%%%%%%%%%%%%%%%%%%%%%%% 下面是正文内容 %%%%%%%%%%%%%%%%%%%%%%%%%%%%%%%%%
%%%%%%%%%%%%%%%%%%%%%%%%%%%%%%%%%%%%%%%%%%%%%%%%%%%%%%%%%%%%%%%%%%%%%%%%%%%%%%%%%

\section{实验目的}

\begin{enumerate}
    \item 对功放及散热片有感性认识,加深对功率放大电路的理解;
    \item 理解功放指标及测量方法;
    \item 初步建立散热和热阻的概念;
    \item 了解电容类别、指标及测试方法;
    \item D 类功放波形及指标测试 (选做)。
    \end{enumerate}
    

\section{实验仪器}

\begin{enumerate}
    \item 数字万用表: Unit UT61E (C190241394)
    \item 数字示波器: RIGOL 200MSO2202A (DS2F192200361)
    \item 信号发生器: GWINSTEK AFG-22225 (GER910370)
    \item 数字直流电源: GWINSTEK GPD-3303S (GES813705)
    \item 其它:LM1875 功放芯片、散热片、导线、电阻、电容等
\end{enumerate}


\section{实验内容及实验结果}

\subsection{依据电路图焊接功放电路}

应注意 LM1875 芯片需要先机械安装再进行焊接,否则芯片难与散热片紧密贴合,会导致散热不良。电路原理图如图 \ref{fig: LM1875 线性功放原理图} 所示,其中 $C_3$, $R_4$, $R_5$ 为 Zobel 网络,用于消振,先不焊接。

\begin{figure}[H]\centering
    \includegraphics[width=0.9\columnwidth]{LCE-03-功率放大器/assets/功率放大器原理图.pdf}
    \caption{LM1875 线性功放原理图}
    \label{fig: LM1875 线性功放原理图}
\end{figure}

\begin{figure}[H]\centering
    \includegraphics[width=0.9\columnwidth]{LCE-03-功率放大器/assets/焊接实物图.png}
    \caption{LM1875 线性功放焊接实物图}
\end{figure}

在接入信号之前,我们需要先进行空载测试,以确认电路没有问题。功放先不接负载,空载测试是否有问题。在连接电源线的时候,多股导线应当铰接在一起,以减少电感和辐射。电源设置为串联模式,两通道电流均设置为 0.1 A 。开启电源,观察电流在 68 mA 附近,电源未进入 Constant Current (恒流) 模式,功放正常工作。


\subsection{观察振荡波形并进行消振}


接入信号源,电源电流设置限流 1.5 A ,在输出端观察信号,发现输出波形震荡较为严重,如下图所示:

\begin{figure}[H]\centering
    \includegraphics[width=\columnwidth]{LCE-03-功率放大器/assets/实验照片/观察输出是否振荡, 无 zobel (input 500mVpp 1kHz).png}
    \caption{Amplifier I/O curve (无 Zobel): 有效输入 (CH1, blue), 输出电压 (CH2, red), 观察输出振荡情况}
\end{figure}

\begin{figure}[H]\centering
    \includegraphics[width=\columnwidth]{LCE-03-功率放大器/assets/实验照片/观察输出是否振荡, 有 zobel (input 500mVpp 1kHz).png}
    \caption{Amplifier I/O curve (有 Zobel): 有效输入 (CH1, blue), 输出电压 (CH2, red), 观察输出消振情况}
\end{figure}

\begin{figure}[H]\centering
    \includegraphics[width=\columnwidth]{LCE-03-功率放大器/assets/实验照片/观察输出是否振荡, 有 zobel (错位显示) (input 500mVpp 1kHz).png}
    \caption{Amplifier I/O curve (有 Zobel): 有效输入 (CH1, blue), 输出电压 (CH2, red), 观察输出消振情况 (错位显示)}
\end{figure}


\newpage
\subsection{测量最大输出摆幅、功率和效率}


我们分别测量了负载为 30 $\Omega$ 和 3.75 $\Omega$ 时的最大输出摆幅、功率和效率。输入信号频率固定为 1 kHz 正弦波,调整输入信号峰峰值以测量最大输出摆幅,结果如下:

\begin{figure}[H]\centering
    \includegraphics[width=\columnwidth]{LCE-03-功率放大器/assets/实验照片/30Ohm输出测试 最大输出摆幅 (测量输入).png}
    \caption{Amplifier I/O curve (30 $\Omega$ load): 有效输入 (CH1, blue), 输出电压 (CH2, red), 测量最大输出摆幅}
\end{figure}

\begin{figure}[H]\centering
    \includegraphics[width=\columnwidth]{LCE-03-功率放大器/assets/实验照片/3R5输出测试 最大输出摆幅 (测量输出).png}
    \caption{Amplifier I/O curve (3.75 $\Omega$ load): 有效输入 (CH1, blue), 输出电压 (CH2, red), 测量最大输出摆幅}
\end{figure}

在每种负载下得到最大输出摆幅后,将输入信号峰峰值降低 5mV,以保证输出在摆幅范围内。然后用万用表测量输入电流和输出电流,计算功率和效率,结果总结在下表:

\begin{table}[H]\centering
    %\renewcommand{\arraystretch}{1.5} % 调整行间距
    %\setlength{\tabcolsep}{1.5mm} % 调整列间距
    \caption{功率放大器数据测试表}
    \label{功率放大器数据测试表}
\begin{tabular}{cccccccccc}\toprule
    负载 $R_{load}$  & 最大输出摆幅 $(V_{out,amp})_{max}$ & 电源 CH1 电压电流 $V_1 @ I_1$ & 电源 CH2 电压电流 $V_2 @ I_2$ \\
    \midrule
    30 $\Omega$ & 27.9 V & 14.995 V @ 0.158 A & 14.999 V @ 0.154 A  \\
    3.75 $\Omega$ & 24.7 V & 14.994 V @ 1.043 A & 14.999 V @ 1.047 A \\
    \bottomrule
\end{tabular}
\end{table}

\begin{table}[H]\centering
    %\renewcommand{\arraystretch}{1.5} % 调整行间距为 1.5 倍
    %\setlength{\tabcolsep}{1.5mm} % 调整列间距
    \caption{功率放大器输出摆幅、功率和效率测试}
    \label{功率放大器输出摆幅、功率和效率测试}
\resizebox{\linewidth}{!}{   % 设置宽度为 \linewidth 等比例缩放
\begin{tabular}{cccccccccc}\toprule
    负载 $R_{load}$ & 最大输出幅度 $(V_{out,amp})_{max}$ & 输出功率 $P_{out,av}$ & 电源功率 $P_{DC,av}$ & 实际效率 $\eta = \frac{P_{out,av}}{P_{DC,av}}$ & 理论效率 $\eta = \frac{\pi V_{out,amp}}{4 V_{CC}}$  \\
    \midrule
    30 $\Omega$ & 13.995 V & 3.2434 W & 4.6791 W & 69.32\,\% & 73.04\,\% \\
    3.75 $\Omega$ & 12.350 V & 20.3363 W & 31.3427 W & 64.88\,\% & 64.66\,\% \\
    \bottomrule
\end{tabular}}
\end{table}


\subsection{扫频法测中频增益和截止频率}

保持负载为 3.75 $\Omega$,开始之前,将输入幅值减小到原来的一半,以保证输出在摆幅范围内。然后设置扫描周期为 600ms ,设定起止频率为 100 Hz 至 10 MHz ,进行扫频。获得的扫频波形如下:


\begin{figure}[H]\centering
    \includegraphics[width=\columnwidth]{LCE-03-功率放大器/assets/实验照片/3R5扫频测试 (输入减半 370 mVpp).png}
    \caption{Amplifier I/O curve (3.75 $\Omega$ load, input 370mVpp): 有效输入 (CH1, blue), 输出电压 (CH2, red), 全局扫频波形}
\end{figure}

分别在中频点、截止频率点将波形展开,然后重新扫频以在此区间获得更高的采样率,得到结果如下:

\begin{figure}[H]\centering
    \includegraphics[width=\columnwidth]{LCE-03-功率放大器/assets/实验照片/3R5扫频测试 中频展开.png}
    \caption{Amplifier I/O curve (3.75 $\Omega$ load, input 370mVpp): 有效输入 (CH1, blue), 输出电压 (CH2, red), 中频点展开}
\end{figure}
\begin{figure}[H]\centering
    \includegraphics[width=\columnwidth]{LCE-03-功率放大器/assets/实验照片/3R5扫频测试 3dB 展开.png}
    \caption{Amplifier I/O curve (3.75 $\Omega$ load, input 370mVpp): 有效输入 (CH1, blue), 输出电压 (CH2, red), 截止频率点展开}
\end{figure}

由上图可知,中频增益和截止频率分别为:
\begin{gather}
A_v = \frac{12.5 \ \mathrm{V}}{370 \ \mathrm{mV}} = 33.7838,\quad 
f_c = 108.2 \ \mathrm{kHz}
\end{gather}

\subsection{测量等效输出电阻}

利用实验二中使用的测量方法,对功放电路输出电阻进行测量,数据与计算结果如下:
\begin{gather}
    R_{out} = \left(\frac{A_0}{A} - 1\right)R_L 
\end{gather}


\begin{table}[H]\centering
    %\renewcommand{\arraystretch}{1.5} % 调整行间距为 1.5 倍
    %\setlength{\tabcolsep}{1.5mm} % 调整列间距
    \caption{等效输出电阻测量数据与计算结果}
    \label{等效输出电阻测量数据与计算结果}
\resizebox{\linewidth}{!}{   % 设置宽度为 \linewidth 等比例缩放
\begin{tabular}{cccccccccc}\toprule
    (空载) 输出电压有效值$(V_{out,RMS})_{R_L = \infty}$ & (负载 3.75 $\Omega$) 输出电压有效值 $(V_{out,RMS})_{R_L = 3.5\Omega}$ & 输出电阻 $R_{out}$  \\
    \midrule
    4.403 V & 4.399 V & 3.410 $\mathrm{m}\Omega$  \\
    \bottomrule
\end{tabular}
}\end{table}



\section{思考题}

\subsection*{4.1 \ \ 低频功放失真度一般要求 1 \%以下,失真度如何测量?}

用示波器测量输出信号的波形,对波形进行傅里叶变换后即可计算出(总)谐波失真度。具体而言,设输出信号中的基波频率为 $f_1$, 基波幅度为 $A_{f_1}$, 其余谐波频率为 $f_2, f_3, \cdots, f_n$, 幅度分别为 $A_{f_2}, A_{f_3}, \cdots A_{f_n}$, 则各谐波对应的失真度(称为谐波失真度)和总谐波失真度 (THD) 定义为:
\begin{gather}
\label{eq: HD}
HD_{f_k} = \frac{A_{f_k}}{A_{f_1}} \times 100\% ,\quad k=2,3,\cdots,n
,\quad 
THD = \sqrt{\sum_{k=2}^{\infty} HD_{f_k}^2} 
= \frac{\sqrt{A_{f_2}^2 + A_{f_3}^2 + \cdots + A_{f_n}^2}}{A_{f_1}} \times 100\%
\end{gather}

下面,我们就来实际测量并计算谐波失真。调整输入信号为 370 mVpp @ 3kHz 以保证输出在摆幅范围内,将示波器测得的波形数据导出,在 MATLAB 中进行数据处理 (MATLAB 源码见附录),可得如下结果:

\begin{figure}[H]\centering
    \includegraphics[width=\columnwidth]{LCE-03-功率放大器/assets/输出波形及其 FT 结果.pdf}
    \caption{谐波失真度测量:输出波形数据及其 FFT (fast Fourier transformer) 结果}
\end{figure}
\begin{figure}[H]\centering
    \includegraphics[width=\columnwidth]{LCE-03-功率放大器/assets/谐波失真图 (线性横坐标).pdf}
    \caption{谐波失真度测量:计算各个谐波的失真度 (线性横坐标)}
\end{figure}
\begin{figure}[H]\centering
    \includegraphics[width=\columnwidth]{LCE-03-功率放大器/assets/谐波失真图 (对数横坐标).pdf}
    \caption{谐波失真度测量:计算各个谐波的失真度 (对数横坐标)}
\end{figure}

依据所得数据,计算出频谱分辨率、各谐波失真度和总谐波失真度如下:
\begin{gather}
\text{
Sampling time: 14 ms, \quad  
sampling rate: 100 MSa/s (受存储深度限制),\quad 
spectral resolution: 71.4286 Hz
}
\end{gather}
\vspace*{-7mm}
\begin{table}[H]\centering
    %\renewcommand{\arraystretch}{1.5} % 调整行间距为 1.5 倍
    %\setlength{\tabcolsep}{1.5mm} % 调整列间距
    \caption{谐波失真度 (HD) 与总谐波失真度 (THD, total harmonic distortion)}
    \label{谐波失真度 (HD) 与总谐波失真度 (THD, total harmonic distortion)}
\resizebox{\linewidth}{!}{   % 设置宽度为 \linewidth 等比例缩放
\begin{tabular}{cccccccccc}\toprule
    谐波失真度 (HD)  & 0.501\,\% @ 6 kHz & 0.169\,\% @ 9 kHz & 0.109\,\% @ 12 kHz & 0.133\,\% @ 15 kHz \\
    \midrule
    总谐波失真度 (THD) & 0.6552\,\% @ 100 kHz & 0.6713\,\% @ 1MHz & 0.7067\,\% @ 10 MHz & 0.8434\,\% @ 100 MHz \\
    \bottomrule
\end{tabular}}
\end{table}




\subsection*{4.2 \ \ 30 欧姆与 3.75 欧姆负载,哪个效率高,为什么?}

负载 30 $\Omega$ 时效率更高,因为此时输出摆幅更大且总输入/输出电流更低。无论是理论近似效率 $\eta  = \frac{\pi V_{out,amp}}{4 V_{CC}}$ 还是实际测得的数据 $\eta_{R_L = 30 \Omega} = 39.32\,\%$, $\eta_{R_L = 3.75 \Omega} = 64.88\,\%$,都说明了负载 30 $\Omega$ 时效率更高。

\subsection*{4.3 \ \ 输出阻抗测量精度较低,如何提高?}

一方面,可以选用阻值更小的负载电阻进行测量,使“分压”现象变得更明显,降低随机误差;另一方面,可以用更精确的测量方式(例如开尔文接线法)消除接触阻抗,降低系统误差。

\subsection*{4.4 \ \ 对本实验及课程有何建议?实验中是否遇到异常现象?}

还是之前提到过的建议:选做实验应给出更详细的内容讲解,以便同学们参考。像这次实验,当我想开始第一个选做实验“D 类功放 TDA8920 测试”时,实在是拔尖四顾心茫然。一方面,课件上并没有给出此电路的原理图或输入输出接口;另一方面,课件上对选择实验的内容描述也不够详细,比如“测量参数”是指什么呢?是功放在某种输入下的功耗、效率,还是其占空比与输入信号的关系,又或一些其它参数?实在是有些摸不着头脑。




















































\newpage
% 附录 A
\newpage
\vspace*{\fill}\begin{center}\Huge{\bfseries 
    附录 A\hspace*{20pt} 预习报告
}\end{center}\addcontentsline{toc}{section}{附录 A\hspace*{6pt} 预习报告} 
\begin{figure}[H]\centering
    \includegraphics[width=\columnwidth]{LCE-03-功率放大器/assets/附录/预习报告.png}
\end{figure}
\vspace*{\fill}
\thispagestyle{fancy} 
\includepdf[pages={-}]{LCE-03-功率放大器/preview/LCE-03 (preview report).pdf}

% 附录 B
\newpage
\begin{center}\Huge{\bfseries 
    附录 B\hspace*{20pt} 原始数据记录表
}\end{center}
\addcontentsline{toc}{section}{附录 B\hspace*{6pt} 原始数据记录表} 
\thispagestyle{fancy} 
\begin{figure}[H]\centering
    \includegraphics[width=0.9\columnwidth]{LCE-03-功率放大器/assets/附录/原始数据.png}
\end{figure}

% 附录
\newpage
\section*{附录 C \hspace*{20pt} Matlab Codes}
\addcontentsline{toc}{section}{附录 C \hspace*{6pt} Matlab Codes} 
\thispagestyle{fancy} 
\lstinputlisting{d:/a_RemoteRepo/GH.MatlabCodes/本科课程代码/Linear Circuit Experiment/LCE_03.m}



\end{document}

% VScode 常用快捷键:

% F2:                       变量重命名
% Ctrl + Enter:             行中换行
% Alt + up/down:            上下移行
% 鼠标中键 + 移动:           快速多光标
% Shift + Alt + up/down:    上下复制
% Ctrl + left/right:        左右跳单词
% Ctrl + Backspace/Delete:  左右删单词    
% Shift + Delete:           删除此行
% Ctrl + J:                 打开 VScode 下栏(输出栏)
% Ctrl + B:                 打开 VScode 左栏(目录栏)
% Ctrl + `:                 打开 VScode 终端栏
% Ctrl + 0:                 定位文件
% Ctrl + Tab:               切换已打开的文件(切标签)
% Ctrl + Shift + P:         打开全局命令(设置)

% Latex 常用快捷键:

% Ctrl + Alt + J:           由代码定位到PDF


