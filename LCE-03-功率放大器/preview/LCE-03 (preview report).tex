% 若编译失败,且生成 .synctex(busy) 辅助文件,可能有两个原因:
% 1. 需要插入的图片不存在:Ctrl + F 搜索 'figure' 将这些代码注释/删除掉即可
% 2. 路径/文件名含中文或空格:更改路径/文件名即可

% --------------------- 文章宏包及相关设置 --------------------- %
% >> ------------------ 文章宏包及相关设置 ------------------ << %
% 设定文章类型与编码格式
\documentclass[UTF8]{article}		
\input{../../.config/config_for_LinearCircuitExperiment.tex}



%%%%%%%%%%%%%%%%%%%%%%%%%%%%%%%%%%%%%%%%%%%%%%%%%%%%%%%%%%%%%%%%
% 仅需修改页眉、实验名称、实验日期
%%%%%%%%%%%%%%%%%%%%%%%%%%%%%%%%%%%%%%%%%%%%%%%%%%%%%%%%%%%%%%%%


%%%%%%%%%%%%%%%%%% 1. 修改页眉内容 %%%%%%%%%%%%%%%%%%
\rhead{Preview Report of LCE-03 功率放大器 (2025.04.18, 丁毅)}

% 开始编辑文章
\begin{document}
\begin{center}\large
    \vspace*{-0.8cm}
    \noindent{\huge\bfseries《\ \ 线\ \ 性\ \ 电\ \ 路\ \ 实\ \ 验\ \ \ 》\ \ 预\ \ 习\ \ 报\ \ 告 }
    \\\vspace{0.1cm}
    \noindent{
    {\bfseries 
%
%%%%%%%%%%%%%%%%%% 2. 修改实验名称 %%%%%%%%%%%%%%%%%%
    实验名称:\uline{\hspace{1.8cm} 功率放大器 \hspace{1.8cm}}
%
    }\hspace{0.4cm}
    指导教师:\uline{\hspace{0.8cm}王东雷\ \ \ \  \ df4dac@sina.com     \hspace{0.8cm}}
    }
    \\\vspace{0.1cm}
    \noindent
    {
    姓名:\uline{\,\,\,丁毅\,\,\,}\hspace{0.2cm}
    学号:\uline{\,\,\,{ 2023K8009908031}\,\,\,}\hspace{0.2cm}
    班级/专业:\uline{\,\,\,{2308/电子信息}\,\,\,}\hspace{0.2cm}
    分组序号:\uline{\,\,\,{2-06}\,\,\,}
    }
    \\\vspace{0.1cm}
    \noindent{
%
%%%%%%%%%%%%%%%%%% 3. 修改实验日期 %%%%%%%%%%%%%%%%%%
    实验日期:\uline{\,\,{2025.04.18}\,\,}\hspace{0.2cm}
%
    实验地点:\uline{\,\,\,教学楼{ 607}\,\,\,}\hspace{0.2cm}
    是否调课/补课:\uline{\hspace{0.26cm}否 \hspace{0.26cm}}\hspace{0.2cm}
    成绩:\uline{\hspace{2cm}}
    }
\end{center}
\vspace{-0.4cm}
\noindent\rule{\textwidth}{0.075em}   % 分割线
\vspace{-1.0cm}


% ------------------------ 文章信息区 ------------------------ %
% ------------------------ 文章信息区 ------------------------ %



%%%%%%%%%%%%%%%%%%%%%%%%%%%%%%%%%%%%%%%%%%%%%%%%%%%%%%%%%%%%%%%%%%%%%%%%%%%%%%%%%
%%%%%%%%%%%%%%%%%%%%%%%%%%%%%%%%% 下面是正文内容 %%%%%%%%%%%%%%%%%%%%%%%%%%%%%%%%%
%%%%%%%%%%%%%%%%%%%%%%%%%%%%%%%%%%%%%%%%%%%%%%%%%%%%%%%%%%%%%%%%%%%%%%%%%%%%%%%%%

\section{实验目的}

\begin{enumerate}
\item 对功放及散热片有感性认识,加深对功率放大电路的理解;
\item 理解功放指标及测量方法;
\item 初步建立散热和热阻的概念;
\item 了解电容类别、指标及测试方法;
\item D 类功放波形及指标测试 (选做)。
\end{enumerate}

\section{实验仪器}

\section{实验内容及步骤}

\begin{enumerate}
    \item 安装焊接LM1875电路,暂时不焊接消振元件C3、R4、R5。
    \item 观察输出是否有振荡,如有则加入 Zobel 网络观察消振效果。
    \begin{itemize}
            \item 测试输出时,电源供电采用 ±15V,设为串联 SER 模式。
            \item 功放先空载测试,电源电流设置为 0.1A,两通道均需设置,开启电源后的电流应在 10 mA  $\sim$ 60 mA 之间;如果电流源进入恒流模式,关闭电路查找原因。
            \item 一切正常后接入信号源观察波形;测试完后连接负载,设置输入电流为1A,开始测试。
    \end{itemize}
    \item 1kHz 下测量输出范围、功率及效率,在 30 $\Omega$ 和 3.75 $\Omega$ 两种负载条件下测试。
    \item 按照实验二中采用的方法测量输出阻抗。
    \item 利用扫频法测量频率响应,幅值取最高输出幅值的一半,只测量高频截止频率。
    \item 接音频信号和扬声器,体会效果。
    \item D类功放:
    \begin{itemize}
        \item 连接电路;
        \item 测量上述参数;
        \item 测量频响;
        \item 测量波形,在信号最高点、零点和最低点处测量芯片半桥输出点波形和占空比,更改电源电压观察占空比变化
    \end{itemize}
\end{enumerate}


\noindent 下面是数据测试表:

\begin{table}[H]\centering
    %\renewcommand{\arraystretch}{1.5} % 调整行间距
    %\setlength{\tabcolsep}{1.5mm} % 调整列间距
    \caption{功率放大器数据测试表}
    \label{功率放大器数据测试表}
\begin{tabular}{cccccccccc}\toprule
    Load & Output Amp. (V) & Effective Amp. (V) & Output Power (W) & Input Power (W) & Efficiency  \\
    \midrule
    30 $\Omega$ &  &   \\
    3.75 $\Omega$ &  &   \\
    \bottomrule
\end{tabular}
\end{table}

\section{同相放大器增益计算}

同相放大器的增益结果如图 \ref{fig: 同相放大器增益计算} 所示;
\begin{gather}
\frac{V_{in,eff.}}{V_{in}} = \frac{R_1}{R_1 + \frac{1}{sC_1}},\quad 
\frac{V_{out}}{V_{in,eff.}} = 1 + \frac{R_2}{R_3 + \frac{1}{sC_2}}
\\ 
\Longrightarrow 
A_v = \frac{V_{out}}{V_{in}} = \frac{R_1}{R_1 + \frac{1}{sC_1}} \cdot \left(1 + \frac{R_2}{R_3 + \frac{1}{sC_2}}\right)
\end{gather}

\begin{figure}[H]\centering
    \begin{subfigure}[b]{0.63\columnwidth}\centering
        \includegraphics[height=150pt]{preview/assets/2025-04-15_16-10-10.pdf}
        \caption{增益计算结果}
    \end{subfigure}
\begin{subfigure}[b]{0.35\columnwidth}\centering
    \includegraphics[height=140pt]{preview/assets/amp.png}
    \caption{原理图}
\end{subfigure}
\caption{同相放大器增益计算}
\label{fig: 同相放大器增益计算}
\end{figure}

\section{注意事项}

\begin{enumerate}
    \item 功放芯片先机械安装,再焊接,否则无法紧密接触散热片;先安装弹性垫片,再安装平垫片;功率原件散热面紧贴散热片,不能有空隙,必要时涂导热硅脂
    \item 3 针接插件应该先组装再焊接;开口朝外,用于接线
    \item 功放 IC 一定要先机械安装再焊接,否则接触不良导致 IC 热关断或拉断引脚 (现象,工作一段时间,输出消失或畸变)
    \item 效率测量出现 $\eta>1$,说明只测量了正电源功率,本实验为双组电源;如果效率测量 $\eta < 0.1$,幅值小,应该在最大幅值条件下测量
    \item 直接测量效率 $\frac{P_{out}}{P_{in}}$ ,不要用输出幅值计算效率
    \item 不要随意用鳄鱼夹,用多股线处理导线端子,避免短路;导线颜色按规定:正电压用红线、负电压用蓝线、地线用白线或黑线,其他线尽量避开这几个颜色
    \item 线路的 GND 应接电源的串联点,而不是电源的 GND 端子,否则会只输出半波信号。
\end{enumerate}


\includepdf[pages={-}]{preview/assets/lm1875.pdf}

%\section{异常现象分析}
















































\end{document}

% VScode 常用快捷键:

% F2:                       变量重命名
% Ctrl + Enter:             行中换行
% Alt + up/down:            上下移行
% 鼠标中键 + 移动:           快速多光标
% Shift + Alt + up/down:    上下复制
% Ctrl + left/right:        左右跳单词
% Ctrl + Backspace/Delete:  左右删单词    
% Shift + Delete:           删除此行
% Ctrl + J:                 打开 VScode 下栏(输出栏)
% Ctrl + B:                 打开 VScode 左栏(目录栏)
% Ctrl + `:                 打开 VScode 终端栏
% Ctrl + 0:                 定位文件
% Ctrl + Tab:               切换已打开的文件(切标签)
% Ctrl + Shift + P:         打开全局命令(设置)

% Latex 常用快捷键:

% Ctrl + Alt + J:           由代码定位到PDF


