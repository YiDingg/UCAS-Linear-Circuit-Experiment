% 若编译失败,且生成 .synctex(busy) 辅助文件,可能有两个原因:
% 1. 需要插入的图片不存在:Ctrl + F 搜索 'figure' 将这些代码注释/删除掉即可
% 2. 路径/文件名含中文或空格:更改路径/文件名即可

% --------------------- 文章宏包及相关设置 --------------------- %
% >> ------------------ 文章宏包及相关设置 ------------------ << %
% 设定文章类型与编码格式
\documentclass[UTF8]{article}		
\input{../../.config/config_for_LinearCircuitExperiment.tex}



%%%%%%%%%%%%%%%%%%%%%%%%%%%%%%%%%%%%%%%%%%%%%%%%%%%%%%%%%%%%%%%%
% 仅需修改页眉、实验名称、实验日期
%%%%%%%%%%%%%%%%%%%%%%%%%%%%%%%%%%%%%%%%%%%%%%%%%%%%%%%%%%%%%%%%


%%%%%%%%%%%%%%%%%% 1. 修改页眉内容 %%%%%%%%%%%%%%%%%%
\rhead{Preview Report of LCE-02 三极管 (2025.04.11, 丁毅)}

% 开始编辑文章
\begin{document}
\begin{center}\large
    \vspace*{-0.8cm}
    \noindent{\huge\bfseries《\ \ 线\ \ 性\ \ 电\ \ 路\ \ 实\ \ 验\ \ \ 》\ \ 预\ \ 习\ \ 报\ \ 告 }
    \\\vspace{0.1cm}
    \noindent{
    {\bfseries 
%
%%%%%%%%%%%%%%%%%% 2. 修改实验名称 %%%%%%%%%%%%%%%%%%
    实验名称:\uline{\hspace{2.2cm} 三极管 \hspace{2.2cm}}
%
    }\hspace{0.4cm}
    指导教师:\uline{\hspace{0.8cm}王东雷\ \ \ \  \ df4dac@sina.com     \hspace{0.8cm}}
    }
    \\\vspace{0.1cm}
    \noindent
    {
    姓名:\uline{\,\,\,丁毅\,\,\,}\hspace{0.2cm}
    学号:\uline{\,\,\,{ 2023K8009908031}\,\,\,}\hspace{0.2cm}
    班级/专业:\uline{\,\,\,{2308/电子信息}\,\,\,}\hspace{0.2cm}
    分组序号:\uline{\,\,\,{2-06}\,\,\,}
    }
    \\\vspace{0.1cm}
    \noindent{
%
%%%%%%%%%%%%%%%%%% 3. 修改实验日期 %%%%%%%%%%%%%%%%%%
    实验日期:\uline{\,\,{2025.04.11}\,\,}\hspace{0.2cm}
%
    实验地点:\uline{\,\,\,教学楼{ 607}\,\,\,}\hspace{0.2cm}
    是否调课/补课:\uline{\hspace{0.26cm}否 \hspace{0.26cm}}\hspace{0.2cm}
    成绩:\uline{\hspace{2cm}}
    }
\end{center}
\vspace{-0.4cm}
\noindent\rule{\textwidth}{0.075em}   % 分割线
\vspace{-1.0cm}


% ------------------------ 文章信息区 ------------------------ %
% ------------------------ 文章信息区 ------------------------ %



%%%%%%%%%%%%%%%%%%%%%%%%%%%%%%%%%%%%%%%%%%%%%%%%%%%%%%%%%%%%%%%%%%%%%%%%%%%%%%%%%
%%%%%%%%%%%%%%%%%%%%%%%%%%%%%%%%% 下面是正文内容 %%%%%%%%%%%%%%%%%%%%%%%%%%%%%%%%%
%%%%%%%%%%%%%%%%%%%%%%%%%%%%%%%%%%%%%%%%%%%%%%%%%%%%%%%%%%%%%%%%%%%%%%%%%%%%%%%%%

\section{Electrical Characteristics pf NPN Transistor SS8050 (宏嘉诚)}

\vspace*{-6mm}
\begin{figure}[H]\centering
    \includegraphics[width=\columnwidth]{preview/assets/SS8050_1.pdf}
    \includegraphics[width=\columnwidth]{preview/assets/SS8050_2.pdf}
\end{figure}\vspace*{-5mm}
\begin{figure}[H]\centering
    \includegraphics[width=\columnwidth]{preview/assets/SS8050_4.pdf}
\end{figure}
\begin{figure}[H]\centering
    \includegraphics[width=\columnwidth]{preview/assets/SS8050_3.pdf}
\end{figure}

\section{Technical Parameters of The Oscilloscope and The Multimeter}

我们的示波器测量范围和精度均高于万用表,因此采用示波器进行测量。示波器 (Rigol 200MSO2202A) 的主要参数如下:
\begin{enumerate}
    \item 带宽: 200 MHz
    \item 输入阻抗::(1 MΩ±1\%)||(16 pF±3 pF)或 50 Ω±1.5\%
    \item 时基档位:1.000 ns/div至1.000 ks/div
    \item 时基精度: ≤±25 ppm
    \item 垂直档位:输入阻抗为50 Ω时:500 μV/div至1 V/div
    \item 输入阻抗为1 MΩ时:500 μV/div至10 V/div
    \item 偏移范围:输入阻抗为50 Ω时: 500 μV/div 至 50 mV/div:±2 V ; 51 mV/div至200 mV/div:±10 V ; 205 mV/div 至1 V/div:±12 V
    \item 输入阻抗为1 MΩ时:500 μV/div至50 mV/div:±2 V ; 51 mV/div至200 mV/div:±10 V ; 205 mV/div至2 V/div:±50 V ; 2.05 V/div至10 V/div:±100 V
    \item 直流增益精度:±2\%满刻度
    \item 直流偏移精度:±0.1 div±2 mV±1\%偏移值
\end{enumerate}

\noindent 万用表 (Unit UT61E) 的主要参数如下:
\begin{enumerate}
    \item 精度:AC 电压测量范围: 220mV/2.2V/22V/220V/750V
    \item AC 电压测量精度:±(0.8\% + 10 digit)
    \item AC 电压测量带宽:45Hz-10kHz
    \item AC 电流量程:220μA/2.2mA/22mA/220mA/10A
    \item AC 电流测量准确度:±(0.8\% + 10 digit)
    \item AC 电流测量带宽:45Hz-10kHz
\end{enumerate}

%\section{异常现象分析}


\section{Common-Emitter Amplifier Design}

\begin{figure}[H]\centering
\begin{subfigure}[b]{0.5\columnwidth}\centering
    \includegraphics[height=190pt]{preview/assets/CE.png}
    \caption{Circuit Schematic}
\end{subfigure}\hfill
\begin{subfigure}[b]{0.5\columnwidth}\centering
    \includegraphics[height=190pt]{preview/assets/Gain.png}
    \caption{Small-Signal (Mid-band) Gain Calculation}
\end{subfigure}
\caption{Design of Common-Emitter Amplifier}
\end{figure}

\section{Input and Output Impedance Calculation}

\noindent 
If considering the coupling capacitors, we have:
\begin{gather}
R_{in} = \frac{1}{j \omega C_1} + R_1 \parallel R_2 \parallel R_{base},\quad 
R_{base} = r_{\pi} + R_E \cdot \frac{\beta r_O + r_O + R_C}{R_E + r_O + R_C}
\\ 
R_{out} = \frac{1}{j \omega C_3} + R_5 \parallel R_{coll},\quad 
R_{coll} = r_O \cdot \left[ 1 + \left(\frac{\beta}{r_{\pi} + R_B} + \frac{1}{r_O}\right)\left(R_E \parallel (r_{\pi} + R_B)\right) \right]
\\
\text{where\ \ }
R_B = R_{Thev} = r_{bb'} + \frac{1}{j \omega C_{1}} \parallel R_1 \parallel R_2
\end{gather}

\noindent 
From another perspective, ignoring the coupling capacitors, we have:
\begin{gather}
R_{in} = R_1 \parallel R_2 \parallel R_{base},\quad 
R_{base} = r_{\pi} + R_E \cdot \frac{\beta r_O + r_O + R_C}{R_E + r_O + R_C}
\\ 
R_{out} = R_5 \parallel R_{coll},\quad 
R_{coll} = r_O \cdot \left[ 1 + \left(\frac{\beta}{r_{\pi} + R_B} + \frac{1}{r_O}\right)\left(R_E \parallel (r_{\pi} + R_B)\right) \right]
\\
\text{where\ \ }
R_B = R_{Thev} = R_{B0} + R_1 \parallel R_2 = r_{bb'}  + R_1 \parallel R_2
\end{gather}
\noindent 
Note that the impedances denote small-signal quantities despite we use the uppercase. For instance, \textbf{ignoring $C_1$ and $C_3$ but considering $C_2$}, and assuming the parameters of the transistor is given by:
\begin{gather}
    I_S = 4.679 \times 10^{-14} \ \mathrm{A}\\
    n_f = 1.01,\ 
    {\color{red} \,\beta = 250\,},\ 
    V_A = 52.64 \ \mathrm{V},\ 
    \\
    R_{B0} = r_{bb'} = 1 \ \Omega,\ 
    R_{E0} = 0.2598 \ \Omega,\ 
    R_{C0} = 1 \ \Omega
\end{gather}
it can be derived that the quiescent operation point is:
\begin{gather}
I_C = 2.178 \ \mathrm{mA},\ 
I_B = 8.712 \ \mathrm{uA}
,\quad 
V_{BE} = 0.644 \ \mathrm{V},\ 
V_{CE} = 2.0547 \ \mathrm{V}
\\
V_E = 0.7651 \ \mathrm{V},\quad 
V_B = 1.4091 \ \mathrm{V},\quad 
V_C = 2.8198 \ \mathrm{V}
\end{gather}
Therefore, the small-signal gain and other parameters is given by (calculated at 1kHz):
\begin{gather}
A_v = -15.6618 - 0.4088j \overset{\mathrm{abs}}{\ =\ } -15.6671,\quad |R_{in}| = 4.8297 \ \mathrm{k}\Omega,\quad |R_{out}| = 992.0509 \ \Omega
\end{gather}

\section{Input/Output Impedance Measurement}

\noindent 
Assuming the open-circuit voltage gain is $A_{0}$, the input source resistance is $R_S$, we have:
\begin{gather}
A_v = \frac{R_{in}}{R_{in} + R_S} A_{0} \Longrightarrow 
R_{in} =  \frac{R_S}{\frac{A_0}{A_v} - 1} \quad (R_L = \infty)
\\ 
A_v = \frac{R_{L}}{R_{L} + R_{out}} A_{0} \Longrightarrow 
R_{out} = \left(\frac{A_0}{A_v} - 1\right) R_L \quad (R_S = 0)
\end{gather}












\section*{Appendix: Matlab Codes of OP, Gain and Impedance Calculation}
\addcontentsline{toc}{section}{附录 \hspace*{6pt} Matlab Codes} 
\thispagestyle{fancy} 
\lstinputlisting{D:/a_RemoteRepo/GH.MatlabCodes/本科课程代码/Linear Circuit Experiment/midterm experiment - common emitter amplifier (20250329).m}





































\end{document}

% VScode 常用快捷键:

% F2:                       变量重命名
% Ctrl + Enter:             行中换行
% Alt + up/down:            上下移行
% 鼠标中键 + 移动:           快速多光标
% Shift + Alt + up/down:    上下复制
% Ctrl + left/right:        左右跳单词
% Ctrl + Backspace/Delete:  左右删单词    
% Shift + Delete:           删除此行
% Ctrl + J:                 打开 VScode 下栏(输出栏)
% Ctrl + B:                 打开 VScode 左栏(目录栏)
% Ctrl + `:                 打开 VScode 终端栏
% Ctrl + 0:                 定位文件
% Ctrl + Tab:               切换已打开的文件(切标签)
% Ctrl + Shift + P:         打开全局命令(设置)

% Latex 常用快捷键:

% Ctrl + Alt + J:           由代码定位到PDF


