% 若编译失败,且生成 .synctex(busy) 辅助文件,可能有两个原因:
% 1. 需要插入的图片不存在:Ctrl + F 搜索 'figure' 将这些代码注释/删除掉即可
% 2. 路径/文件名含中文或空格:更改路径/文件名即可

% --------------------- 文章宏包及相关设置 --------------------- %
% >> ------------------ 文章宏包及相关设置 ------------------ << %
% 设定文章类型与编码格式
\documentclass[UTF8]{article}		
\input{../../.config/config_for_LinearCircuitExperiment.tex}



%%%%%%%%%%%%%%%%%%%%%%%%%%%%%%%%%%%%%%%%%%%%%%%%%%%%%%%%%%%%%%%%
% 仅需修改页眉、实验名称、实验日期
%%%%%%%%%%%%%%%%%%%%%%%%%%%%%%%%%%%%%%%%%%%%%%%%%%%%%%%%%%%%%%%%


%%%%%%%%%%%%%%%%%% 1. 修改页眉内容 %%%%%%%%%%%%%%%%%%
\rhead{Preview Report of LCE-05 精密整流 (2025.05.09, 丁毅)}

% 开始编辑文章
\begin{document}
\begin{center}\large
    \vspace*{-0.8cm}
    \noindent{\huge\bfseries《\ \ 线\ \ 性\ \ 电\ \ 路\ \ 实\ \ 验\ \ \ 》\ \ 预\ \ 习\ \ 报\ \ 告 }
    \\\vspace{0.1cm}
    \noindent{
    {\bfseries 
%
%%%%%%%%%%%%%%%%%% 2. 修改实验名称 %%%%%%%%%%%%%%%%%%
    实验名称:\uline{\hspace{0.7cm} 精密整流与有效值检测 \hspace{0.7cm}}
%
    }\hspace{0.4cm}
    指导教师:\uline{\hspace{0.8cm}王东雷\ \ \ \  \ df4dac@sina.com     \hspace{0.8cm}}
    }
    \\\vspace{0.1cm}
    \noindent
    {
    姓名:\uline{\,\,\,丁毅\,\,\,}\hspace{0.2cm}
    学号:\uline{\,\,\,{ 2023K8009908031}\,\,\,}\hspace{0.2cm}
    班级/专业:\uline{\,\,\,{2308/电子信息}\,\,\,}\hspace{0.2cm}
    分组序号:\uline{\,\,\,{2-06}\,\,\,}
    }
    \\\vspace{0.1cm}
    \noindent{
%
%%%%%%%%%%%%%%%%%% 3. 修改实验日期 %%%%%%%%%%%%%%%%%%
    实验日期:\uline{\,\,{2025.05.09}\,\,}\hspace{0.2cm}
%
    实验地点:\uline{\,\,\,教学楼{ 607}\,\,\,}\hspace{0.2cm}
    是否调课/补课:\uline{\hspace{0.26cm}否 \hspace{0.26cm}}\hspace{0.2cm}
    成绩:\uline{\hspace{2cm}}
    }
\end{center}
\vspace{-0.4cm}
\noindent\rule{\textwidth}{0.075em}   % 分割线
\vspace{-1.0cm}


% ------------------------ 文章信息区 ------------------------ %
% ------------------------ 文章信息区 ------------------------ %



%%%%%%%%%%%%%%%%%%%%%%%%%%%%%%%%%%%%%%%%%%%%%%%%%%%%%%%%%%%%%%%%%%%%%%%%%%%%%%%%%
%%%%%%%%%%%%%%%%%%%%%%%%%%%%%%%%% 下面是正文内容 %%%%%%%%%%%%%%%%%%%%%%%%%%%%%%%%%
%%%%%%%%%%%%%%%%%%%%%%%%%%%%%%%%%%%%%%%%%%%%%%%%%%%%%%%%%%%%%%%%%%%%%%%%%%%%%%%%%

\section{实验目的}

\begin{enumerate}
    \item 设计由运放构成的精密整流电路,测量波形、精度及带宽;
    \item 加深对有效值定义的理解,了解交流信号检测的方法;
    \item 测试有效值检测电路 AD367,在不同波形下与精密整流电路对比;
    \item 练习使用面包板。
\end{enumerate}
\section{实验仪器}

\begin{enumerate}
    \item 数字万用表: Unit UT61E (C190241394)
    \item 数字示波器: RIGOL 200MSO2202A (DS2F192200361)
    \item 信号发生器: GWINSTEK AFG-22225 (GER910370)
    \item 数字直流电源: GWINSTEK GPD-3303S (GES813705)
    \item 多功能数字测量仪: 
    % 这里是链接
    \href{https://digilent.com/reference/test-and-measurement/analog-discovery/start
    }{ % 这里是文字
    Analog Discovery 1
    } 
    (D704387)
    \item 其它:面包板,运算放大器 LF412,电容、电阻、二极管 (1N4148, 1N5711)、导线等
\end{enumerate}


\section{实验内容概要}

我们并没有直接使用课件上的电路,而是参考了 TI 的应用电路 % 这里是链接
\href{https://www.ti.com/lit/ug/tidu030/tidu030.pdf
}{ % 这里是文字
TI Verified Design: Precision Full-Wave Rectifier, Dual-Supply {\color{black} (https://www.ti.com/lit/ug/tidu030/tidu030.pdf)}
},完整电路如下图所示:

\begin{figure}[H]\centering
    \includegraphics[width=0.7\columnwidth]{preview/assets/image.png}
    \caption{TI verified design: precision full-wave rectifier, dual-supply}
    \label{TI Verified Design: Precision Full-Wave Rectifier, Dual-Supply}
\end{figure}

\begin{enumerate}
\item 开始实验之前,调整 AD1 Scope 功能的 Plot Width = 4;
\item 用运放和 Diode 1 (1N4148) 搭建 TI 的精密整流电路,如图 \ref{TI Verified Design: Precision Full-Wave Rectifier, Dual-Supply} 所示;
\item 用 AD1 的 Impedance 功能测量整流电路的频率响应 (input 50 mVamp, 500 mVamp, 5 Vamp),{\color{red} 保存图像和数据} (数据导出后在 MATLAB 作图,标出最高工作频率);
\item 用 AD1 的 Scope 功能测量整流电路在低频 (50 Hz) 和最高工作频率处的输入输出波形,{\color{red} 保存图像,保存 50 Hz 时的输入输出数据 (用于思考题)};
\item 用 AD1 的 Tracer 功能,在 50 Hz 下,改变输入信号幅度,测量输出信号幅度,以观察输出幅度的关于输入幅度的线性性,{\color{red} 保存图像};
\item 对比整流电路和 AD637 在不同输入信号下的响应:(a) 正弦波, (b) 三角波, (c) Burst 正弦波 (D = 0.1, 1ms/10ms), {\color{red} 保存三个图像};
\item 将二极管替换为 diode 2 (1N5711),重复步骤 (3) $\sim$ (4);
\item (选做) 搭建 Peak Detector (峰值检测电路),用 AD1 的 Scope 功能测量输入输出波形,保存图像。
\end{enumerate}

\section{精密半波整流波形}

第一个运放在正半周导通,反相输出,输入输出波形如下图:

\begin{figure}[H]\centering
    \includegraphics[width=\columnwidth]{preview/assets/image copy.png}
    \caption{Input and output waveforms of the first op-amp}
\end{figure}

%\section{异常现象分析}
















































\end{document}

% VScode 常用快捷键:

% F2:                       变量重命名
% Ctrl + Enter:             行中换行
% Alt + up/down:            上下移行
% 鼠标中键 + 移动:           快速多光标
% Shift + Alt + up/down:    上下复制
% Ctrl + left/right:        左右跳单词
% Ctrl + Backspace/Delete:  左右删单词    
% Shift + Delete:           删除此行
% Ctrl + J:                 打开 VScode 下栏(输出栏)
% Ctrl + B:                 打开 VScode 左栏(目录栏)
% Ctrl + `:                 打开 VScode 终端栏
% Ctrl + 0:                 定位文件
% Ctrl + Tab:               切换已打开的文件(切标签)
% Ctrl + Shift + P:         打开全局命令(设置)

% Latex 常用快捷键:

% Ctrl + Alt + J:           由代码定位到PDF


