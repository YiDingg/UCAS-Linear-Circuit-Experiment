% 若编译失败,且生成 .synctex(busy) 辅助文件,可能有两个原因:
% 1. 需要插入的图片不存在:Ctrl + F 搜索 'figure' 将这些代码注释/删除掉即可
% 2. 路径/文件名含中文或空格:更改路径/文件名即可

% --------------------- 文章宏包及相关设置 --------------------- %
% >> ------------------ 文章宏包及相关设置 ------------------ << %
% 设定文章类型与编码格式
\documentclass[UTF8]{article}		
\input{../../.config/config_for_LinearCircuitExperiment.tex}



%%%%%%%%%%%%%%%%%%%%%%%%%%%%%%%%%%%%%%%%%%%%%%%%%%%%%%%%%%%%%%%%
% 仅需修改页眉、实验名称、实验日期
%%%%%%%%%%%%%%%%%%%%%%%%%%%%%%%%%%%%%%%%%%%%%%%%%%%%%%%%%%%%%%%%


%%%%%%%%%%%%%%%%%% 1. 修改页眉内容 %%%%%%%%%%%%%%%%%%
\rhead{Preview Report of LCE-04 场效应管 (2025.04.25, 丁毅)}

% 开始编辑文章
\begin{document}
\begin{center}\large
    \vspace*{-0.8cm}
    \noindent{\huge\bfseries《\ \ 线\ \ 性\ \ 电\ \ 路\ \ 实\ \ 验\ \ \ 》\ \ 预\ \ 习\ \ 报\ \ 告 }
    \\\vspace{0.1cm}
    \noindent{
    {\bfseries 
%
%%%%%%%%%%%%%%%%%% 2. 修改实验名称 %%%%%%%%%%%%%%%%%%
    实验名称:\uline{\hspace{2.0cm} 场效应管 \hspace{2.0cm}}
%
    }\hspace{0.4cm}
    指导教师:\uline{\hspace{0.8cm}王东雷\ \ \ \  \ df4dac@sina.com     \hspace{0.8cm}}
    }
    \\\vspace{0.1cm}
    \noindent
    {
    姓名:\uline{\,\,\,丁毅\,\,\,}\hspace{0.2cm}
    学号:\uline{\,\,\,{ 2023K8009908031}\,\,\,}\hspace{0.2cm}
    班级/专业:\uline{\,\,\,{2308/电子信息}\,\,\,}\hspace{0.2cm}
    分组序号:\uline{\,\,\,{2-06}\,\,\,}
    }
    \\\vspace{0.1cm}
    \noindent{
%
%%%%%%%%%%%%%%%%%% 3. 修改实验日期 %%%%%%%%%%%%%%%%%%
    实验日期:\uline{\,\,{2025.04.25}\,\,}\hspace{0.2cm}
%
    实验地点:\uline{\,\,\,教学楼{ 607}\,\,\,}\hspace{0.2cm}
    是否调课/补课:\uline{\hspace{0.26cm}否 \hspace{0.26cm}}\hspace{0.2cm}
    成绩:\uline{\hspace{2cm}}
    }
\end{center}
\vspace{-0.4cm}
\noindent\rule{\textwidth}{0.075em}   % 分割线
\vspace{-1.0cm}


% ------------------------ 文章信息区 ------------------------ %
% ------------------------ 文章信息区 ------------------------ %



%%%%%%%%%%%%%%%%%%%%%%%%%%%%%%%%%%%%%%%%%%%%%%%%%%%%%%%%%%%%%%%%%%%%%%%%%%%%%%%%%
%%%%%%%%%%%%%%%%%%%%%%%%%%%%%%%%% 下面是正文内容 %%%%%%%%%%%%%%%%%%%%%%%%%%%%%%%%%
%%%%%%%%%%%%%%%%%%%%%%%%%%%%%%%%%%%%%%%%%%%%%%%%%%%%%%%%%%%%%%%%%%%%%%%%%%%%%%%%%

\section{实验目的}

\begin{enumerate}
    \item 加深对 FET 的理解;
    \item 测量 FET 转移特性;
    \item 搭建 MOSFET 放大电路,与双极型晶体管对比;
\end{enumerate}


\section{实验仪器}

\begin{enumerate}
    \item 数字万用表: Unit UT61E (C190241394)
    \item 数字示波器: RIGOL 200MSO2202A (DS2F192200361)
    \item 信号发生器: GWINSTEK AFG-22225 (GER910370)
    \item 数字直流电源: GWINSTEK GPD-3303S (GES813705)
    \item 多功能数字测量仪: 
    % 这里是链接
    \href{https://digilent.com/reference/test-and-measurement/analog-discovery/start
    }{ % 这里是文字
    Analog Discovery 1
    } 
    (D704387)
    \item 晶体管测试板: % 这里是链接
    \href{https://yidingg.github.io/YiDingg/#/ElectronicDesigns/Simplified\%20Transistor\%20Tester
    }{ % 这里是文字
    Simplified Transistor Tester
    }
    \item 其它: 3DJ7H (N-Channel JFET)、2N7000 (N-Channel VDMOS)、 IRFP460 (N-Channel Power MOS)、电容、电阻、导线、跳线、测试点等
\end{enumerate}

\section{实验内容概要}

\begin{enumerate}
\item 用万用表测量 FET 的等效二极管压降
\item 焊接 MOSFET 放大器 PCB 板
\item 测量 MOSFET 静态特性曲线及转移特性曲线,测试方法详见 % 这里是链接
    \href{https://yidingg.github.io/YiDingg/\#/Blogs/Electronics/Transistor\%20Measurement\%20Methods
    }{ % 这里是文字
    Transistor Measurement Methods
    };测试完成后,对所得数据进行处理,计算出 $r_O$、$g_m$、$\frac{g_m}{I_D}$ (transconductance efficiency) 等小信号参数以及 $R_{ON}$ (导通电阻);
\item 测量 common-source amplifier 的波形 (要有图片)、增益曲线 (100 Hz $\sim$ 1 MHz)、输入输出阻抗 (100 Hz $\sim$ 1 MHz),增益与阻抗曲线的测量需要用到 Analog Discovery 1 (后简称``AD1'');增益曲线可在测输出阻抗的 $A_1$ 时测得,无需重复测量;
\item 测量 common-drain amplifier (source follower) 的波形 (要有图片)、增益曲线 (100 Hz $\sim$ 1 MHz)、输入输出阻抗 (100 Hz $\sim$ 1 MHz),增益及阻抗曲线的测量需要用到 AD1;增益曲线可在测输出阻抗的 $A_1$ 时测得,无需重复测量;
\item 更改跳线,测量 CS 组态开关波形;
\item (选做) 测量 JFET 静态特性曲线及转移特性曲线,测试方法及步骤同第 (3) 条。
\end{enumerate}

\section{输入输出阻抗的理论与实验测量公式}

MOSFET 三种基本放大器的输入输出阻抗理论值如表 \ref{Three basic types of CMOS amplifiers} 所示,其中 $R_{drain}$ 和 $R_{source} $ 电阻的含义是:
\begin{gather}
R_{D0} = r_O,\quad  R_{S0} = \frac{1}{g_m}\parallel \frac{1}{g_{mb}}\parallel r_O
\\
R_{drain} = \left(1 + \frac{R_S}{R_{S0}}\right)R_{D0},\quad 
R_{source} = \left(1 + \frac{R_D}{r_O}\right)R_{S0}
\end{gather}


\begin{table}[H]\centering
    \renewcommand{\arraystretch}{1.1} % 调整行间距
    %\setlength{\tabcolsep}{1.5mm} % 调整列间距
    \caption{Three basic types of CMOS amplifiers}
    \label{Three basic types of CMOS amplifiers}
    \begin{tabular}{cccccccccc}\toprule
        Parameter & CS (Common Source) & CD (SF, Source Follower) & CG (Common Gate) \\
    \midrule $R_{out}$ 
        & $ \displaystyle R_{D} \parallel R_{drain}$ 
        & $ \displaystyle R_S \parallel R_{source}$
        & $ \displaystyle R_D \parallel R_{drain}$
    \\ $ \displaystyle G_{m}$ 
    & $ \frac{g_m}{1 + \frac{R_S}{R_{S0}}} $ 
    & $ \frac{-g_m}{1 + \frac{R_D}{r_O}}$
    & $ \frac{-1}{R_S + R_{S0}}$
    \\ $ \displaystyle  A_v$ 
    & $  -g_m r_O \cdot \frac{R_D}{R_D + R_{drain}}$ 
    & $ g_m R_{S0} \cdot \frac{R_S}{R_S + R_{source}}$
    & $ \frac{R_D \parallel R_{drain}}{R_S + R_{source}}$
    \\\midrule
    $ \displaystyle \lim_{r_O \to \infty} R_{out}$ 
        & $ R_D$ 
        & $ R_{S} \parallel \frac{1}{g_m} \parallel \frac{1}{g_{mb}}$
        & $ R_D$
    \\ $ \displaystyle \lim_{r_O \to \infty} G_{m}$ 
        & $ \frac{1}{(1 + \eta) R_S + \frac{1}{g_m}}$ 
        & $ - g_m$
        & $ \frac{-1}{R_S + \frac{1}{g_m + g_{mb}}}$
    \\ $ \displaystyle \lim_{r_O \to \infty} A_v$ 
        & $ \frac{- R_D}{(1+\eta) R_S + \frac{1}{g_m} }$ 
        & $ \frac{R_S}{(1 + \eta)R_S + \frac{1}{g_m } }$
        & $ \frac{R_D}{R_S + \frac{1}{(1+\eta)g_m} }$
    \\\midrule
    $ \displaystyle \lim_{\substack{g_{mb} \to 0 \\ r_O \to \infty}} R_{out}$ 
        & $ R_D$ 
        & $ R_{S} \parallel \frac{1}{g_m}$
        & $ R_D$
    \\ $ \displaystyle \lim_{\substack{g_{mb} \to 0 \\ r_O \to \infty}} G_{m}$ 
        & $ \frac{g_m}{1 + g_m R_S}$ 
        & $ - g_m$
        & $ \frac{-1}{R_S + \frac{1}{g_m}} = \frac{- g_m}{1 + g_m R_S}$
    \\ $ \displaystyle \lim_{\substack{g_{mb} \to 0 \\ r_O \to \infty}} A_v$ 
        & $ \frac{- R_D}{R_S + \frac{1}{g_m}}$ 
        & $ \frac{R_S}{R_S + \frac{1}{g_m}}$
        & $ \frac{R_D}{R_S + \frac{1}{g_m}}$
    \\
        \bottomrule
    \end{tabular}
\end{table}

\noindent
设 $A_1$ 为实验测得的原始增益,$A_2$ 为加入特定电阻后的增益,则有计算公式:
\begin{gather}
Z_{in} = \frac{R_S}{\left(\frac{A_1}{A_2} - 1\right)},\quad 
Z_{out} = \left(\frac{A_1}{A_2} - 1\right) R_L
\end{gather}
注意 $A_1$ 和 $A_2$ 是复数,当两者相位区别不大时,可作近似:
\begin{gather}
|Z_{in}| \approx \frac{R_S}{\left(\left|\frac{A_1}{A_2}\right| - 1\right)},\quad 
|Z_{out}| \approx \left(\left|\frac{A_1}{A_2}\right| - 1\right) R_L
\end{gather}



\section{Electrical Characteristics of N-Channel VDMOS 2N7000 (onsemi)}
\includepdf[pages={-}]{preview/assets/2N7000 (onsemi).PDF}
%\section{异常现象分析}
















































\end{document}

% VScode 常用快捷键:

% F2:                       变量重命名
% Ctrl + Enter:             行中换行
% Alt + up/down:            上下移行
% 鼠标中键 + 移动:           快速多光标
% Shift + Alt + up/down:    上下复制
% Ctrl + left/right:        左右跳单词
% Ctrl + Backspace/Delete:  左右删单词    
% Shift + Delete:           删除此行
% Ctrl + J:                 打开 VScode 下栏(输出栏)
% Ctrl + B:                 打开 VScode 左栏(目录栏)
% Ctrl + `:                 打开 VScode 终端栏
% Ctrl + 0:                 定位文件
% Ctrl + Tab:               切换已打开的文件(切标签)
% Ctrl + Shift + P:         打开全局命令(设置)

% Latex 常用快捷键:

% Ctrl + Alt + J:           由代码定位到PDF


